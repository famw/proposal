\chapter{\label{chap:lit-review}Revisão Bibliográfica}

\section{SpelunkBots}

TBD

\subsection{Spelunky}
% Genero
% Objetivos
% Regras do jogo
% Exemplos de mapa

TBD

\subsection{GameMaker}

GameMaker é uma plataforma para a criação de jogos que traz consigo uma série
de ferramentas que facilitam o trabalho do desenvolvedor de jogos. Contando com
funcionalidades como editores de \textit{scripts}\footnote{Código desenvolvido
para o controle dos comportamentos dos elementos do jogo} e de \textit{sprites}
\footnote{Elementos visuais do jogo, tais como o personagem, o fundo, os
inimigos. Representados como uma ou mais imagens, permitindo que as mesmas
sejam animadas},
gerenciadores de eventos e outras \cite{GMAKER8DOCS}, o GameMaker oferece um
ótimo suporte ao desenvolvedor para a criação de jogos.

\subsection{Competição SpelunkBots}
% O framework
% A competição

TBD

\section{Agentes Racionais}

Um agente pode ser visto como todo e qualquer tipo de entidade que seja capaz
de perceber o ambiente onde está situado, através de seus sensores, podendo
executar ações nesse ambiente conforme sua necessidade através de seus atuadores.
\cite{Russell:1995:AIM:193191}

Para exemplificar, podemos tomar como exemplo o ser humano, que consegue
perceber o ambiente através de seus olhos e ouvidos, por exemplo, conseguindo
agir no ambiente com suas partes do corpo, como braços, pernas, mãos e etc.
\cite{Russell:1995:AIM:193191}

Nesse contexto, um agente racional é um agente que toma as ações que o levam a
ser o mais bem-sucedido no que o mesmo está proposto.

\subsection{Tipos de agentes}
% Reativos, pró-ativos, sociais

TBD

\subsection{Ambiente}

TBD

\subsection{Objetivo}

TBD

\subsection{Estado}

TBD

\subsection{Agentes BDI}

TBD

\section{Frameworks de Inteligência Artificial}

TBD

\subsection{Motivação}

TBD

\subsection{Exemplos de Frameworks e Competições}

TBD

\section{Planejamento}

TBD

\section{Aprendizado de Máquina}

TBD

\subsection{Q-Learning}

TBD

\subsection{SARSA}

TBD

\subsection{Deep Learning}

TBD
