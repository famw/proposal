\chapter{\label{chap:problem}O Problema}
Estabelecer a complexidade computacional de jogos -- sejam jogos de carta, jogos
de tabuleiro ou jogos digitais -- é uma prática muito importante e comum.
Através desta categorização, obtemos evidências do porquê humanos consideram
estes problemas como sendo interessantes, além de indicar para pesquisadores da
área os desafios propostos vistos de uma perspectiva de tarefa de otimização.
Dados os desafios presentes em Spelunky concluiu-se que, computacionalmente
falando, trata-se de, no mínimo, um problema do conjunto \textit{NP-Hard}
\cite{SPELUNKYHARD}.

O ambiente em Spelunky é \textbf{contínuo}, \textbf{parcialmente observável},
\textbf{dinâmico}, \textbf{estocástico} e \textbf{sequencial}. Estas
características influenciam fortemente na dificuldade do problema proposto pelo
jogo. Somado a isto, os níveis são gerados proceduralmente, cortando a
possibilidade de memorização do mapa. Contudo, o algoritmo utilizado para gerar
os níveis garante que existe pelo menos um caminho transponível do início ao fim
-- mesmo que com inimigos e armadilhas no caminho --, sem que seja necessário o
uso de bombas ou cordas para ajudar com desobstrução e locomoção. Alguns
exemplos de geração de mapa podem ser observados a seguir:

\begin{figure}[htb!]
\centering\includegraphics[width=.65\textwidth]{fig/spelunky-levels-example.png}
\caption {\label{fig:spelunkbots-debug-screen}Exemplos de níveis gerados
proceduralmente.} \end{figure}

% diversas acoes e combinacoes de acoes possiveis

\begin{itemize}
    \item Construção de um bot capaz de jogar uma partida de Spelunky
    \item Explicar e detalhar o ambiente onde o jogador estará inserido
    \item Explicar como o jogador pode interagir com o ambiente
    \item Enumerar as possíveis ações do jogador
    \item Explicar os cenários onde o jogador pode perder vidas
    \item Explicar como o bot pode ``ver'' o ambiente
    \item Detalhar a API do Spelunkbots
\end{itemize}
