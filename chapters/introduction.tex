\chapter{\label{chap:intro}Introdução}
A popularização dos jogos digitais se deu nas décadas de 70 e 80 com o
surgimento dos computadores pessoais, os fliperamas e os \textit{consoles} de
jogos. Apesar de sofrer uma retração com o \textit{crash} de 83
\footnote{Recessão na indústria de jogos digitais que ocorreu de 1983 a 1985. A
principal causa foi a saturação do mercado.}, o mercado de jogos digitais
ascendeu rapidamente. Em 2014, obteve uma receita de aproximadamente 80 bilhões
de dólares\footnote{https://newzoo.com/insights/articles/top-100-countries-represent-99-6-81-5bn-global-games-market/}.
Isto mostra que os seres humanos possuem um interesse significativo por jogos
digitais. Como consequência, a indústria está em constante aprimoramento,
buscando produzir jogos mais interessantes, inovadores e desafiadores, a fim de
manter seu público alvo e atrair novos jogadores.

Paralelamente, a área de inteligência artificial vê em jogos digitais um grande
potencial para servir de plataforma de testes para seus algoritmos e técnicas.
Alguns dos motivos são seu baixo custo -- se comparado, por exemplo, a área da
robótica --, sua facilidade de acesso e sua capacidade de reproduzir cenários
reais em um ambiente virtual.

Diversas competições de inteligência artificial foram criadas recentemente
utilizando jogos digitais como plataforma de teste \cite{GameAiCompetition}.
Estas competições geralmente possuem como objetivo principal a criação de
agentes inteligentes capazes de jogar o jogo proposto da maneira mais eficiente
possível, seja por pontuação, seja por tempo. Uma destas competições, baseada no
jogo de computador Spelunky, se chama SpelunkBots. Os criadores desta competição
desenvolveram uma aplicação de inteligência artificial que facilita o
desenvolvimento de agentes para o jogo.

Spelunky é um jogo de computador considerado fácil de aprender mas extremamente
difícil de dominar, pois requer que o jogador possua reações rápidas para
perigos iminentes e, ao mesmo tempo, pense em táticas e estratégias de longo
prazo para ser bem-sucedido. Ao todo, são 26 inimigos, 14 armadilhas e 43 itens
diferentes \footnote{Dados extraídos da Spelunky Wiki
(http://spelunky.wikia.com/wiki/Spelunky\_Wiki).}. O jogador, ao iniciar uma
partida, sempre se depara com um nível completamente diferente, pois o jogo faz
uso de um algoritmo complexo para gerar níveis. Estes fatores tornam Spelunky um
jogo com desafios interessantes para a pesquisa em inteligência artificial.

Este trabalho tem como objetivo a construção de agentes inteligentes capazes de
jogar o jogo Spelunky, considerado como um problema de difícil solução. Para
tal, desejamos estudar as técnicas de inteligência artificial mais utilizadas
para a criação de agentes inteligentes em jogos digitais e aplicar este
conhecimento em conjunto da aplicação SpelunkBots.
