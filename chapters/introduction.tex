\chapter{\label{chap:intro}Introdução}

Muitos seres humanos têm um interesse especial por jogos, sejam jogos
esportivos, jogos de tabuleiro, jogos de carta e, mais recentemente, jogos
digitais. Tal interesse faz com que esses jogos tenham que ser cada vez mais
desafiadores, especialmente tratando-se de jogos digitais, que a cada ano que
passa trazem cada vez mais inovação e elementos que visam atraír e manter o
jogador.

Por outro lado, antes da criação de muitos jogos digitais, pesquisadores da área
de inteligência artifical voltavam seus estudos para como resolver problemas
utilizando computadores, desses problemas, muitos envolvem melhorar a vida dos
seres humanos, seja através de alguma forma de poupar trabalho, ou então,
tomando decisões complexas, por exemplo.

Assim, o compartilhamento de conhecimento estre essas duas áreas traz uma
relação mutuamente benéfica. Para a área de inteligência artifical, permite que
muitos conceitos sejam estudados e aplicados em um ambiente que traz uma série
de desafios a serem tratados. Para a área de jogos, utilizar elementos de
inteligência artifical traz um dinamismo que pode garantir o interesse de um
jogador por um determinado jogo.

\begin{enumerate}
    \item Estudos na área de IA permitem a solução de problemas antes
        impossíveis ou difíceis de solucionar
    \item Crescimento do mercado de jogos digitais e a necessidade de trazer
        mais ``realismo'' para esses jogos, utilizando inteligência artificial
        como apoio
    \item O jogo Spelunky é um jogo considerado difícil e desafiador, exigindo
        reações rápidas e estratégias a longo prazo, devido ao número de
        armadilhas inimigos que o jogador encontra no caminho. E através de um 
        framework de inteligência artifical criado para o mesmo -- o
        SpelunkyBots -- traz desafios para pesquisadores na área de inteligência
        artificial e jogos digitais.
    \item O presente trabalho apresenta a proposta de construção de bots para o
        jogo.
    \item O presente trabalho contribui com as áreas de inteligência artificial
        e jogos digitais, aplicando os conceitos e técnicas da primeira para
        obtenção de sucesso na segunda.
\end{enumerate}
