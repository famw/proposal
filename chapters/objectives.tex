\chapter{\label{chap:objectives}Objetivos}
Este trabalho tem como objetivo geral a criação de um \textit{bot} que seja
capaz de jogar uma partida de Spelunky da melhor maneira possível. As métricas
de avaliação utilizadas serão pontuação e tempo. Para auxiliar nesta tarefa,
utilizar-se-á o \textit{framework} SpelunkBots. Esta ferramenta comunica-se
diretamente com o jogo, possibilitando a aquisição de informações do ambiente -- como localização de inimigos, armadilhas e tesouros --, o envio de ações para o explorador, entre outras facilidades -- algumas mencionadas anteriormente. Após estudo mais aprofundado das técnicas de inteligência artifical citadas neste documento, selecionaremos as que mais se adequarem para a resolução do problema proposto para incorporar durante o desenvolvimento do projeto.

% detalhar objetivos específicos (TC I)

\begin{itemize}
    \item Objetivo Geral
        \subitem Desenvolver uma inteligência artificial capaz de jogar uma partida de Spelunky
    \item Objetivos Específicos
        \subitem Entender o jogo Spelunky
        \subitem Estudar o \textit{framework} SpelunkBots
        \subitem Estudar e escolher os métodos de IA mais adequados para auxiliar na solução do problema
\end{itemize}
