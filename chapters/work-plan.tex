\chapter{\label{chap:work-plan}Etapas do Trabalho}

A execução do trabalho particiona-se em iterações de uma semana, onde serão
desenvolvidos um ou mais objetivos específicos. A tabela a seguir ilustra o
cronograma definido para a primeira etapa do projeto, com base nos objetivos
definidos no capítulo \ref{chap:objectives}:

\begin{table}[htb!]
\centering
\caption{Objetivos por Iteração}
\label{tab:work-plan}
\begin{tabular}{c|c|c|c|c|c|c|c|c|c|c|}
\cline{2-11}
{\bf}                                 & \multicolumn{3}{c|}{{\bf Abr/16}} & \multicolumn{4}{c|}{{\bf Mai/16}}     & \multicolumn{3}{c|}{{\bf Jun/16}} \\ \hline
\multicolumn{1}{|c|}{{\bf Objetivo}}  & {\bf 1} & {\bf 2} & {\bf 3}	     & {\bf 4} & {\bf 5} & {\bf 6} & {\bf 7} & {\bf 8} & {\bf 9} & {\bf 10}      \\ \hline
\multicolumn{1}{|c|}{{\bf 1}}         & X       & X       & X             &         &         &         &         &         &         &               \\ \hline
\multicolumn{1}{|c|}{{\bf 2}}         &         &         &               & X       &         &         &         &         &         &               \\ \hline
\multicolumn{1}{|c|}{{\bf 3}}         &         &         &               &         & X       &         &         &         &         &               \\ \hline
\multicolumn{1}{|c|}{{\bf 4}}         &         &         &               &         &         &  X      & X       &         &         &               \\ \hline
\multicolumn{1}{|c|}{{\bf 5}}         &         &         &               &         &         &         & X       & X       &         &               \\ \hline
\multicolumn{1}{|c|}{{\bf 6}}         &         &         &               &         &         &         &         &         & X       & X             \\ \hline
\end{tabular}
\end{table}

\section{Detalhamento dos Objetivos}

\begin{enumerate}
    \item Estudar o jogo Spelunky e o \textit{framework} SpelunkBots
    \begin{description}[leftmargin=!,labelwidth=\widthof{\bfseries Descrição}]
        \item [Descrição]
			Obtenção de maior conhecimento sobre os desafios impostos pelo
			jogo; entendimento das mecânicas e controles de jogo, padrões de
			ataque de inimigos e funcionamento das armadilhas;  estudo do
			algoritmo de geração de níveis utilizado; obtenção de maior
			conhecimento sobre detalhes de funcionamento do SpelunkBots; estudo
			dos \textit{bots} e níveis disponibilizados de exemplo de uso da
			\textit{API}; detecção das limitações e restrições impostas pelo
			uso do SpelunkBots.
        \item [Iterações]
            1, 2 e 3
        \item [Período]
            11/04/2016 a 02/05/2016
    \end{description}

    \item Estudar agentes BDI
    \begin{description}[leftmargin=!,labelwidth=\widthof{\bfseries Descrição}]
        \item [Descrição]
			Pesquisa e leitura de artigos científicos sobre agentes BDI;
			pesquisa e leitura de conteúdo sobre uso da técnica para criação de
			criação de agentes em jogos digitais; levantamento de ferramentas
			que	facilitem o uso da técnica; realizar uma modelagem do uso de
			agentes BDI com o SpelunkBots.
        \item [Iterações]
            4
        \item [Período]
            02/05/2016 a 09/05/2016
    \end{description}
	
	\item Estudar planejamento
    \begin{description}[leftmargin=!,labelwidth=\widthof{\bfseries Descrição}]
        \item [Descrição]
            Pesquisa e leitura de artigos científicos sobre planejamento;
			pesquisa e leitura de conteúdo sobre uso da técnica para criação de
			criação de agentes em jogos digitais; levantamento de ferramentas
			que	facilitem o uso da técnica; realizar uma modelagem do uso de
			planejamento com o SpelunkBots.
        \item [Iterações]
            5
        \item [Período]
            09/05/2016 a 16/05/2016
    \end{description}

	\item Estudar aprendizado de máquina
    \begin{description}[leftmargin=!,labelwidth=\widthof{\bfseries Descrição}]
        \item [Descrição]
        Pesquisa e leitura de artigos científicos sobre aprendizado de máquina;
		pesquisa e leitura de conteúdo sobre uso da técnica para criação de
		criação de agentes em jogos digitais; levantamento de ferramentas
		que	facilitem o uso da técnica; realizar uma modelagem do uso de
		aprendizado de máquina com o SpelunkBots.
		\item [Iterações]
            6 e 7
        \item [Período]
            16/05/2016 a 30/05/2016
    \end{description}

	\item Estudar redes neurais.
    \begin{description}[leftmargin=!,labelwidth=\widthof{\bfseries Descrição}]
        \item [Descrição]
        Pesquisa e leitura de artigos científicos sobre redes neurais;
		pesquisa e leitura de conteúdo sobre uso da técnica para criação de
		criação de agentes em jogos digitais; levantamento de ferramentas
		que	facilitem o uso da técnica; realizar uma modelagem do uso de
		redes neurais com o SpelunkBots.
		\item [Iterações]
            7 e 8
        \item [Período]
            23/05/2016 a 06/06/2016
    \end{description}

	\item Selecionar técnicas de inteligência artificial para auxiliar na
	solução do problema
    \begin{description}[leftmargin=!,labelwidth=\widthof{\bfseries Descrição}]
        \item [Descrição]
			Apuração do conhecimento obtido sobre as técnicas selecionadas;
			seleção de uma técnica ou conjunto de técnicas que auxiliem no
			desenvolvimento de \textit{bots} para o jogo Spelunky.
        \item [Iterações]
            9 e 10
        \item [Período]
            06/06/2016 a 20/06/2016
    \end{description}

\end{enumerate}

