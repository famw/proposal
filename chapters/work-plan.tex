\chapter{\label{chap:work-plan}Etapas do Trabalho}

A execução do trabalho particiona-se em iterações de uma semana, onde serão
desenvolvidos um ou mais objetivos específicos. A tabela a seguir ilustra o
cronograma definido para a primeira etapa do projeto, com base nos objetivos
definidos no capítulo \ref{chap:objectives}:

\begin{table}[htb!]
\centering
\caption{Objetivos por Iteração}
\label{tab:work-plan}
\begin{tabular}{c|c|c|c|c|c|c|c|c|c|c|}
\cline{2-11}
{\bf}                                 & \multicolumn{3}{c|}{{\bf Abr/16}} & \multicolumn{4}{c|}{{\bf Mai/16}}     & \multicolumn{3}{c|}{{\bf Jun/16}} \\ \hline
\multicolumn{1}{|c|}{{\bf Objetivo}}  & {\bf 1} & {\bf 2} & {\bf 3}       & {\bf 4} & {\bf 5} & {\bf 6} & {\bf 7} & {\bf 8} & {\bf 9} & {\bf 10}      \\ \hline
\multicolumn{1}{|c|}{{\bf 1}}         & X       & X       & X             &         &         &         &         &         &         &               \\ \hline
\multicolumn{1}{|c|}{{\bf 2}}         &         &         &               & X       & X       & X       &         &         &         &               \\ \hline
\multicolumn{1}{|c|}{{\bf 3}}         &         &         &               &         &         &         & X       & X       & X       & X             \\ \hline
\end{tabular}
\end{table}

\section{Detalhamento dos Objetivos}

\begin{enumerate}
    \item Estudar o jogo Spelunky
    \begin{description}[leftmargin=!,labelwidth=\widthof{\bfseries Descrição}]
        \item [Descrição]
            Obtenção de maior conhecimento sobre os desafios impostos pelo
            jogo; entendimento das mecânicas e controles de jogo, padrões de
            ataque de inimigos e funcionamento das armadilhas; compreensão da
            utilidade dos itens e equipamentos, bem como as sinergias entre
            eles; estudo do algoritmo de geração de níveis utilizado;
            elaboração de táticas que auxiliem a vencer o jogo;
        \item [Iterações]
            1, 2 e 3
        \item [Período]
            11/04/2016 a 02/05/2016
    \end{description}

    \item Estudar o \textit{framework} SpelunkBots
    \begin{description}[leftmargin=!,labelwidth=\widthof{\bfseries Descrição}]
        \item [Descrição]
            Obtenção de maior conhecimento sobre detalhes de funcionamento do
            \textit{framework}; estudo da \textit{API} fornecida; estudo dos
            \textit{bots} e níveis disponibilizados de exemplo de uso da
            \textit{API}; detecção das limitações e restrições impostas pelo
            uso desta tecnologia.
        \item [Iterações]
            4, 5 e 6
        \item [Período]
            02/05/2016 a 23/05/2016
    \end{description}

    \item Estudar técnicas de inteligência artificial para auxiliar na solução
    do problema.
    \begin{description}[leftmargin=!,labelwidth=\widthof{\bfseries Descrição}]
        \item [Descrição]
            Pesquisa e leitura de artigos científicos que aplicaram conceitos e
            técnicas de inteligência artificial a jogos digitais; levantamento
            de técnicas de inteligência artificial que auxiliem na construção
            de \textit{bots} para jogos digitais; escolha da técnica ou
            combinação de técnicas mais indicada para o desenvolvimento dos
            \textit{bots} para Spelunky.
        \item [Iterações]
            7, 8, 9 e 10
        \item [Período]
            23/05/2016 a 20/06/2016
    \end{description}

\end{enumerate}

